\documentclass[12pt]{article}
\topmargin -0.7in
\oddsidemargin -0.21in
\evensidemargin -0.21in
\textwidth=17cm
\textheight=25cm

\usepackage[utf8]{inputenc}
\usepackage[T2A]{fontenc} 
\usepackage[russian]{babel}
\usepackage{amsmath}
\usepackage{comment}

\usepackage{tikz}
\usepackage{graphicx}
\graphicspath{{./}}
\DeclareGraphicsExtensions{.png}

\begin{document}
Обычный текст здесь

\begin{tikzpicture}
% оси координат
\draw[->] (-2, 0) coordinate (O) -- +(5, 0) node [right] {$z$};
\draw[->] (O) -- +(0, 7) node [left] {$t$};
% пунктирные прямые
\draw[thick, dashed]
(0, 0) coordinate (A) -- +(-45:.5) 
(0, 0) -- +(-135:.5)
(0, 0) -- ++(45:3) -- ++(135:2) coordinate (B) -- ++(225:3) -- +(315:2) ++(45:3)
-- ++(45:3) -- ++(135:2) coordinate (C) -- ++(225:3) -- +(315:2) ++(45:3)
 -- +(45:.5) 
 +(0,0) -- +(135:.5);
% гиперболы
 \draw[thick]
(A) .. controls +(50:2.8) .. (B)
(A) .. controls +(130:1.85) .. (B)
(B) .. controls +(50:2.8) .. (C)
(B) .. controls +(130:1.85) .. (C)
(A) -- +(-55:.4)
(A) -- +(-125:.4)
(C) -- +(125:.4)
(C) -- +(55:.4);
\end{tikzpicture}

И здесь
\end{document}
